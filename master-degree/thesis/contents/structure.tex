% !TEX root = ../thesis.tex
\chapter{学位论文基本结构}
学位论文基本结构包括前置部份、主体部份和结尾部份\footnote{测试脚注另起一章编号的变化}。
\section{前置部分包括}
\begin{enumerate}
	\item 封面
	\item 题名页
	\item 英文题名页(硕士可省略)
	\item 独创性声明(知识产权声明?)
	\item 勘误表(可根据需要)
	\item 致谢
	\item 序言或前沿(可根据需要)
	\item 摘要页
	\item 目次页
	\item 插图和附表清单(可根据需要)
	\item 缩写、符号清单、术语表(可根据需要)
\end{enumerate}
\section{主体部分}
\begin{enumerate}
	\item 引言(绪论)
	\item 正文
	\item 结论
\end{enumerate}
\section{结尾部分}
\begin{enumerate}
	\item 参考文献
	\item 附录(可根据需要)
	\item 索引(根据需要)
	\item 作者简历及在学期间所取得的科研成果
	\item 封底
\end{enumerate}
\chapter{版面设置}
\section{字体设置}
字体设置
\begin{table}[htb]
	\caption{文章字体设置效果}
	\label{tab:文章字体设置效果}
	\begin{center}
		\begin{tabular}{ccc}
			\toprule
					& 英文字体 & 中文字体  \\
			\midrule
			正文字体 & I can eat glass, it doesn't hurt me. & 我能吞下玻璃而不伤身体 \\
			\textbackslash textrm\{\} & \textrm{I can eat glass, it doesn't hurt me.} & \textrm{我能吞下玻璃而不伤身体} \\
			\textbackslash textsf\{\} & \textsf{I can eat glass.} & \textsf{我能吞下玻璃而不伤身体} \\
			\textbackslash texttt\{\} & \texttt{I can eat glass.} & \texttt{我能吞下玻璃而不伤身体} \\
			\textbackslash textbf\{\} & \textbf{I can eat glass.} & \textbf{我能吞下玻璃而不伤身体} \\
			\bottomrule
		\end{tabular}
	\end{center}
\end{table}
