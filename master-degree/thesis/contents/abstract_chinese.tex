% !TEX root = ../thesis.tex

% 定义中文摘要和关键字
\begin{cabstract}
请注意,以下内容是参考自\textbf{薛瑞尼}的清华大学论文模板,主要是为了填内容方便。

论文的摘要是对论文研究内容和成果的高度概括。摘要应对论文所研究的问题及其研究目
的进行描述,对研究方法和过程进行简单介绍,对研究成果和所得结论进行概括。摘要应
具有独立性和自明性,其内容应包含与论文全文同等量的主要信息。使读者即使不阅读全
文,通过摘要就能了解论文的总体内容和主要成果。

论文摘要的书写应力求精确、简明。切忌写成对论文书写内容进行提要的形式,尤其要避
免“第 1 章……;第 2 章……;……”这种或类似的陈述方式。

本文介绍浙江大学论文模板的使用方法。本模板符合学校的硕士、博士论文格式要求。
写这个模板的主要原因是想深入学习一下\LaTeX,还有可以自己毕业的时候用。

论文的摘要是对论文研究内容和成果的高度概括。摘要应对论文所研究的问题及其研究目
的进行描述,对研究方法和过程进行简单介绍,对研究成果和所得结论进行概括。摘要应
具有独立性和自明性,其内容应包含与论文全文同等量的主要信息。使读者即使不阅读全
文,通过摘要就能了解论文的总体内容和主要成果。

论文摘要的书写应力求精确、简明。切忌写成对论文书写内容进行提要的形式,尤其要避
免“第 1 章……;第 2 章……;……”这种或类似的陈述方式。

本文介绍浙江大学论文模板的使用方法。本模板符合学校的硕士、博士论文格式要求。
写这个模板的主要原因是想深入学习一下\LaTeX,还有可以自己毕业的时候用。




论文摘要的书写应力求精确、简明。切忌写成对论文书写内容进行提要的形式,尤其要避
免“第 1 章……;第 2 章……;……”这种或类似的陈述方式。

本文介绍浙江大学论文模板的使用方法。本模板符合学校的硕士、博士论文格式要求。
写这个模板的主要原因是想深入学习一下\LaTeX,还有可以自己毕业的时候用。

本文的创新点主要有:
\begin{itemize}
	\item 用例子来解释模板的使用方法;
	\item 用废话来填充无关紧要的部分;
	\item 一边学习摸索一边编写新代码。
\end{itemize}

关键词是为了文献标引工作、用以表示全文主要内容信息的单词或术语。关键词不超过 5
个,每个关键词中间用分号分隔。(模板作者注:关键词分隔符不用考虑,模板会自动处
理。英文关键词同理。)
\end{cabstract}

\ckeywords{\TeX, \LaTeX, CJK, 模板, 毕业论文}
